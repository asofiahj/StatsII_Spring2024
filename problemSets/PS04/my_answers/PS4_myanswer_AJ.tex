\documentclass[oneside]{article}
\usepackage{graphicx} % Required for inserting images
\usepackage[margin=1in]{geometry}
\setlength{\parindent}{0pt}
\setlength{\parskip}{0.5em}

\title{Problem Set 4}
\author{Ana Sofia Jesus 19327602}
\date{07/04/2024}

\begin{document}

\maketitle

\section*{Question 1}

After loading the necessary packages and importing the data from the eha library, the Cox proportional hazard model was estimated:
\begin{verbatim}
cox_model <- coxph(Surv(enter, exit, event) ~ m.age + sex, data = child)
\end{verbatim}

The results obtained are displayed on Table 1. 

\begin{table}[!htbp] \centering 
  \caption{Coefficients for Cox Proportional Hazard model} 
  \label{} 
\begin{tabular}{@{\extracolsep{5pt}}lc} 
\\[-1.8ex]\hline 
\hline \\[-1.8ex] 
 & \multicolumn{1}{c}{\textit{Dependent variable:}} \\ 
\cline{2-2} 
\\[-1.8ex] & enter \\ 
\hline \\[-1.8ex] 
 m.age & 0.008$^{***}$ \\ 
  & (0.002) \\ 
  & \\ 
 sexfemale & $-$0.082$^{***}$ \\ 
  & (0.027) \\ 
  & \\ 
\hline \\[-1.8ex] 
Observations & 26,574 \\ 
R$^{2}$ & 0.001 \\ 
Max. Possible R$^{2}$ & 0.986 \\ 
Log Likelihood & $-$56,503.480 \\ 
Wald Test & 22.520$^{***}$ (df = 2) \\ 
LR Test & 22.518$^{***}$ (df = 2) \\ 
Score (Logrank) Test & 22.530$^{***}$ (df = 2) \\ 
\hline 
\hline \\[-1.8ex] 
\textit{Note:}  & \multicolumn{1}{r}{$^{*}$p$<$0.1; $^{**}$p$<$0.05; $^{***}$p$<$0.01} \\ 
\end{tabular} 
\end{table} 
 
The estimated coefficient for the mother's age (m.age) is positive (0.007617), suggesting that for each one-unit increase in mother's age, the logged hazard ratio increases approximately 0.76\%, while holding infant gender constant.
\\
This effect of m.age on the hazard is statistically significant at a high level of confidence (0.001).
\\
On the other hand, the estimated coefficient for female gender (compared to male gender) is negative (-0.082215), suggesting that being female is associated with a lower hazard compared to being male, while holding mother's age constant.
\\
In other words, being female is associated with a hazard approximately 7.89\% lower (1 - 0.9211 = 0.0789 decrease) compared to being male, for mothers of the same age.
\\
This effect of gender on the mortality hazard is also statistically significant at the level of confidence (0.01).
\\
These interpretations provide insights into the factors influencing historical child mortality rates based on mother's age and infant's gender in Sweden from 1850 to 1884. 

\end{document}
